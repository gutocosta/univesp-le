\documentclass[a4paper]{article}
\usepackage[utf8]{inputenc}


% Pacote que define o layout da lista de exercícios
% Opções:
% professor   Exibe as respostas e as resoluções (PADRÃO)
% mediador    Exibe apenas as resoluções. Deve ser usado para gerar a versão da lista para os moderadores.
% aluno       Exibe apenas as respostas. Deve ser usado para gerar a versão da lista para os alunos.
\usepackage[mediador]{univesp-le}
%\usepackage{paralist}

% REMOVER. Usado apenas para gerar texto lipsum
\usepackage{lipsum}

% Define o nome da disciplina, do professor-autor e as aulas às quais a lista de exercício refere-se
\disciplina{Física II} % = \title
\autor{Ivan Ramos Pagnossin}           % = \author
\aulas{1--2}

\begin{document}

% Imprime o nome da disciplina, do professor-autor e as aulas às quais a lista de exercício refere-se
\maketitle

% Cada ambiente exercicio inicia um novo exercício, independentemente de tratar-se do enunciado, resposta ou resolução.
\begin{exercicio*}
  Este é o enunciado de um exercício do portfólio, pois foi criado com o ambiente \texttt{exercicio*} (com asterisco).
\end{exercicio*}

\begin{exercicio}
  \label{ex:qquer}
  Enunciado de um exercício ordinário, criado com o ambiente \texttt{exercicio} (sem asterisco).
  \textbf{\MakeUppercase{Negrito}}.\footnote{Uma nota de rodapé.}
\end{exercicio}

\begin{exercicio}
  \begin{enumerate}
  \item Enunciado de um exercício ordinário, criado com o ambiente \texttt{exercicio} (sem asterisco).
  \footnote{Outra nota de rodapé.}
  \item Item de exercício.
  \item Outro item de exercício.
  \end{enumerate}
\end{exercicio}

\begin{exercicio}
  Este exercício faz referência ao exercício~\ref{ex:qquer}.
\end{exercicio}

\begin{exercicio}
  \begin{enumerate}
  \item Item de exercício.
  \item Outro item de exercício.
  \end{enumerate}
  
  \begin{inlineenum}
    Este é um teste de itens de exercício inline, onde os itens aparecem em sequência:
    \inlineitem Um item;
    \inlineitem outro item;
    \inlineitem terceiro item.
  \end{inlineenum}

\end{exercicio}

\begin{exercicio}
  \lipsum[1-20]
\end{exercicio}

\begin{exercicio}
  Exercício curto.
  O \LaTeX\ tentará \emph{não} quebrar a página entre ``EXERCÍCIO X'' e o texto que o sucede, mas isso nem sempre é possível.
\end{exercicio}

\begin{exercicio}
  Exercício curto.
\end{exercicio}

\begin{exercicio}
  Exercício curto.
\end{exercicio}

\begin{exercicio}
  Exercício curto.
\end{exercicio}

\begin{exercicio}
  Exercício curto.
\end{exercicio}

\begin{exercicio}
  Exercício curto.
\end{exercicio}

% Use o ambiente respostas para as respostas.
\begin{respostas}
  \begin{exercicio*}
    Resposta de um exercício do portfólio:
    \emph{não} aparece para o aluno;
    mas aparece para o mediador e para o professor.
  \end{exercicio*}
  
  \begin{exercicio}
    Resposta de um exercício ordinário.
  \end{exercicio}
  
  \begin{exercicio}
    \lipsum[1-18]
  \end{exercicio}
  
  \begin{exercicio}
    Exercício curto.
  \end{exercicio}

  \begin{exercicio}
    Exercício curto.
  \end{exercicio}

  \begin{exercicio}
    Exercício curto.
  \end{exercicio}

  \begin{exercicio}
    Exercício curto.
  \end{exercicio}

  \begin{exercicio}
    Exercício curto.
  \end{exercicio}

  \begin{exercicio}
    Exercício curto.
  \end{exercicio}
\end{respostas}

% Use o ambiente resolucoes para as resoluções.
\begin{resolucoes}
  \begin{exercicio*}
    Resolução de um exercício do portfólio.
    No caso das resoluções, não há diferença entre exercícios ordinários e do portfólio.
  \end{exercicio*}
  
  \begin{exercicio}
    Resolução de um exercício ordinário.
  \end{exercicio}
  
  \begin{exercicio}
    \lipsum[1-20]
  \end{exercicio}
  
 \begin{exercicio}
    Exercício curto.
  \end{exercicio}

  \begin{exercicio}
    Exercício curto.
  \end{exercicio}

  \begin{exercicio}
    Exercício curto.
  \end{exercicio}

  \begin{exercicio}
    Exercício curto.
  \end{exercicio}

  \begin{exercicio}
    Exercício curto.
  \end{exercicio}

  \begin{exercicio}
    Exercício curto.
  \end{exercicio}

\end{resolucoes}

\end{document}
