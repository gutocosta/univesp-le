\documentclass[a4paper]{article}
\usepackage[utf8]{inputenc}

% Fonte sem serifa com suporte a expressões matemáticas
\usepackage[sfmath]{kpfonts}
\usepackage[T1]{fontenc}

% Pacote que define o layout da lista de exercícios
% Opções:
% professor   Exibe as respostas e as resoluções (PADRÃO)
% mediador    Exibe apenas as resoluções. Deve ser usado para gerar a versão da lista para os moderadores.
% aluno       Exibe apenas as respostas. Deve ser usado para gerar a versão da lista para os alunos.
\usepackage[aluno]{univesp-le}

% REMOVER. Usado apenas para gerar texto lipsum
\usepackage{lipsum}

% Define o nome da disciplina, do professor-autor e as aulas às quais a lista de exercício refere-se
\disciplina{Física II para Engenharia} % = \title
\autor{Ivan Ramos Pagnossin}           % = \author
\aulas{1--2}

\begin{document}

% Imprime o nome da disciplina, do professor-autor e as aulas às quais a lista de exercício refere-se
\maketitle
 
% Cada \exercicio inicia um novo exercício, independentemente de tratar-se do enunciado, resposta ou resolução.
\exercicio \lipsum[1-2]
\exercicio \lipsum[3]
\exercicio Teste teste
\exercicio Teste teste
\exercicio Teste teste
\exercicio \lipsum[4]
\exercicio \lipsum[5-8]
\exercicio \lipsum[10]

% Use o ambiente respostas para as respostas.
\begin{respostas}
  \exercicio \lipsum[1-2]
  \exercicio \lipsum[3]
  \exercicio \lipsum[4]
  \exercicio \lipsum[5-8]
  \exercicio Teste teste
  \exercicio Teste teste
  \exercicio Teste teste
  \exercicio \lipsum[10]
\end{respostas}

% Use o ambiente resolucoes para as resoluções.
\begin{resolucoes}
  \exercicio \lipsum[1-2]
  \exercicio \lipsum[3]
  \exercicio \lipsum[4]
  \exercicio \lipsum[5-8]
  \exercicio \lipsum[9]
  \exercicio Teste teste
  \exercicio Teste teste
  \exercicio Teste teste
\end{resolucoes}

\end{document}
