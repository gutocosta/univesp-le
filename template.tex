% -----------------------------------------------------------------------------------------
% Encoding: UTF-8
% -----------------------------------------------------------------------------------------
% Modelo de lista de exercício da Universidade Virtual do Estado de São Paulo (Univesp),
% usando o pacote univesp-le, e instruções de uso do pacote.
% Autor:    Ivan Ramos Pagnossin
% Data:     Janeiro de 2015
% Idioma:   pt-BR
% Licença:  Creative Commons BY-SA International 4.0
%
% Template of a list of exercise of Universidade Virtual do Estado de São Paulo (Univesp),
% which uses package univesp-le, and instructions of use.
% Author:   Ivan Ramos Pagnossin
% Date:     January 2015
% Language: pt-BR
% License:  Creative Commons BY-SA International 4.0
% -----------------------------------------------------------------------------------------
\documentclass[a4paper]{article}
\usepackage[utf8]{inputenc}

% -----------------------------------------------------------------------------------------
% O pacote univesp-le encarrega-se de cuidar de quase todos os detalhes de
% layout das listas de exercícios da Univesp. Ele tem apenas uma opção:
% 
% aluno   	Gera a versão da lista que vai para o aluno:
%               ela contém os enunciados e as respostas, mas
%               oculta as respostas dos exercícios indicados
%               para o portfólio.
%
% Além disso, o pacote define uma série de comandos para criar a
% lista de exercícios apropriadamente:
%
% Comandos \disciplina
%          \autor
%          \aulas
%          \inlineitem
%
% Ambientes exercicio
%           exercicio*
%           respostas
%           resolucoes
%           inlineenum
%
% Veja abaixo como usá-los.
% -----------------------------------------------------------------------------------------
\usepackage{univesp-le} % <-- Gera a versão do moderador/professor
%\usepackage[aluno]{univesp-le} % <-- Gera a versão do aluno

\disciplina{Nome da disciplina}
\autor{Nome do professor-autor}
\aulas{XX--YY} % Aulas às quais essa lista de exercício está associada.

  
  
  
  
  
  
  
  
% Usado apenas neste modelo.
\newcommand\code[1]{\textbf{#1}}

\begin{document}

  
  
  
  
  
  
  
  
  % ------------- ENUNCIADOS -------------
  % O ambiente exercicio é usado para criar o enunciado, a resposta e a resolução de um exercício.
  \begin{exercicio}
    Este é o enunciado de um exercício.
    \begin{enumerate}
      \item Um item de exercício.
      \item Outro item de exercício.
    \end{enumerate}
  \end{exercicio}

  
  
  % Use o ambiente exercicio* (com asterisco) para indicar um exercício que vai para o portfólio do aluno.
  \begin{exercicio*}\label{ex:qquer}%
    Este é o enunciado de um exercício do portfólio, pois foi criado com o ambiente \code{exercicio*} (com asterisco).
    Perceba a anotação ``(portfólio)'' ao lado da numeração do exercício.
  \end{exercicio*}

  
  
  \begin{exercicio}
    Este exercício faz uma referência ao exercício \ref{ex:qquer}.
    Ele também tem itens, mas desta vez eles seguem a linha normalmente:%
    \begin{inlineenum}
      \inlineitem Um item de exercício.
      \inlineitem Outro item de exercício.
    \end{inlineenum}

  \end{exercicio}

  
  
  
  
  
  
  
  
  % ------------- RESPOSTAS -------------
  % Use o ambiente respostas para agrupar as respostas dos exercícios.
  % É isso que distingue as respostas dos enunciados.
  \begin{respostas}
    
    O ``gabarito'' contém as \emph{respostas} dos exercícios e sempre é exibido.
    
    Se a opção \code{aluno} for utilizada,
    as respostas dos exercícios indicados para o portfólio serão ocultas.
    No lugar delas aparecerá o texto padronizado
    ``Resposta não disponível: este exercício faz parte do portfólio''.
    
    \begin{exercicio}
      Aqui vem a resposta do exercício.
    \end{exercicio}
    
  
  
    \begin{exercicio*}
      Esta é a resposta do exercício que vai para o portfólio, pois como o seu enunciado, foi criada com o ambiente \code{exercicio*} (com asterisco).
    \end{exercicio*}

  
  
    \begin{exercicio}
      Note ainda que VOCÊ precisa cuidar para que as respostas estejam na ordem correta,
      para que a numeração delas corresponda à dos enunciados.
    \end{exercicio}

  \end{respostas}

  
  
  
  
  
  
  
  
  % ------------- RESOLUÇÕES -------------
  % Use o ambiente resolucoes para agrupar as respostas dos exercícios.
  % É isso que distingue as resoluções dos enunciados.
  \begin{resolucoes}

    O ``gabarido do moderador'' contém as \emph{resoluções} dos exercícios.
    Se a opção \code{aluno} for usada, ele será removido do documento.

    \begin{exercicio}
      Aqui vem a resolução do exercício.
    \end{exercicio}
    
  
  
    \begin{exercicio*}
      Esta é a resolução do exercício que vai para o portfólio.
      Como no caso das respostas, os exercícios indicados para o portfólio serão anotados com ``(portfólio)'' após a numeração do exercício.
    \end{exercicio*}

  
  
    \begin{exercicio}
      Note ainda que VOCÊ precisa cuidar para que as resoluções estejam na ordem correta,
      para que a numeração delas corresponda à dos enunciados.
    \end{exercicio}

  \end{resolucoes}
  
  
  
  
  
  
  
  
  
\end{document}
